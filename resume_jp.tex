% LaTeX file for resume 
% This file uses the resume document class (res.cls)

\documentclass{res} 
%\usepackage{helvetica} % uses helvetica postscript font (download helvetica.sty)
%\usepackage{newcent}   % uses new century schoolbook postscript font 
\newsectionwidth{0pt}  % So the text is not indented under section headings
\usepackage{fancyhdr}  % use this package to get a 2 line header
\usepackage{hyperref}
\renewcommand{\headrulewidth}{0pt} % suppress line drawn by default by fancyhdr
\setlength{\headheight}{24pt} % allow room for 2-line header
\setlength{\headsep}{24pt}  % space between header and text
\setlength{\headheight}{24pt} % allow room for 2-line header
\pagestyle{fancy}     % set pagestyle for document
% \rhead{ {\it Z. Zinger}\\{\it p. \thepage} } % put text in header (right side)
\cfoot{}                                     % the foot is empty
\topmargin=-0.5in % start text higher on the page

\hypersetup{
  colorlinks=true,
  linkcolor=blue,
  filecolor=magenta,
  urlcolor=cyan,
}

\begin{document}
\thispagestyle{empty} % this page has no header  
\name{JULIAN GRINBLAT\\[12pt]} % the \\[12pt] adds a blank line after name

\address{\begin{tabular}{l}{\bf 住所} \\ % for some reason since the second address is a table, the first one has to be too to get the alignment correctly
   146-0085, 東京都, 大田区  \\
   久が原 3-30-9, コリーヌ久が原104号室 \\
   (090) 7768-6902
\end{tabular}}

\address{\begin{tabular}{@{}ll}
   Email: & \href{mailto:julian@dotcore.co.il}{julian@dotcore.co.il} \\
   GitHub: & \url{https://github.com/perrin4869} \\
   NPM: & \url{https://www.npmjs.com/~perrin4869}
\end{tabular}}

\begin{resume}
 
\section{\centerline{職歴}} 
\vspace{8pt}
{\sl 楽天株式会社} \hfill        2015年4月 ~ 現在
\vspace{8pt}

\title{APPLICATION ENGINEER}
\employer{}
\location{}
\dates{2018年1月 ~ 現在}
\begin{position}
セルフサービス環境のフロントエンドポータルに務めていた。
当初はReactコンポーネントのみとして開発されたものであったが、
多数のライブラリーを導入し、より良い構築やテストにつなげることができた。
使用したライブラリは以下の通りである: redux、react-final-form、ramda、normalizr,
reselect、re-reselect、mocha、chai、sinon、postcss、等。\\

 \begin{itemize} \itemsep -2pt % reduce space between items
   \item Reduxを利用し、アプリケーションのGlobal Stateを管理した。
ミドルウェアを使って、APIからの情報読み取り、Local Storageでのデータ
を保存等の副作用処理を分離させた。reselectとre-reselectでセレクター
を書いた。
  \item eslintとairbnbルールでコードのlintingを実施した。
  \item mocha、chai、sinon、enzymeに基づいたテストを構築した。
  \item react-final-formを導入して、フォームのstateを管理した。
  \item PostCSSをwebpackで実行することで、ポータルのCSSのビルド
することができた。BootstrapやFont Awesome、CSSライブラリーの管理がより
簡単に行うことができた。
  \item webpackとimageminで画像の圧縮がすることができた。
 \end{itemize}
\end{position}

\title{DEVOPS ENGINEER}
\employer{}
\location{}
\dates{2015年6月 ~ 2017年12月}
\begin{position}
新卒研修後、サーバープラットフォームチームに配属された。様々なストレージソリューション(Pure Storage, Nimble Storage等)や、
OpenStack環境、NuageのSDN環境の管理を務めた。
担当したタスクは:\\

\begin{itemize} \itemsep -2pt % reduce space between items
  \item Fluentd、InfluxDB、Grafana、SNMP、ParamikoやPythonのスクリプトで
ストレージソリューションのレーテンシーやIOPS、テレメトリデータのダッシュボード
を構築した。
  \item PythonスクリプトでSDNのオペレーションの自動化。バックアップやVMware Clusterの
資源の配分、定常オペレーションはより簡単かつ正確になった。VMwareをpyVmomiで操作した。
  \item OpenStack環境でsysctl変数がリブート後に消える問題を解決。Ansibleのplaybook
を適切に編集して、必要なノードに変更を施した。
  \item OpenStack環境のVMをテナント交換した。MySQLレコードを手動で変更するやりかたを
調べ、テスト環境で試した。マニュアルを書いて、オペレーションを無事に実行した。
\end{itemize}
\end{position}

\vspace{0.2in} 
\section{\centerline{学歴}} 
\vspace{8pt} 
{\sl 学士(理学)}, 物理 \\ % \sl will be bold italic in
大阪大学 理学部物理学科      \hfill    2015年3月
  
\vspace{0.2in} 

\newpage

\section{\centerline{オープンソース コントリビューション}}
\vspace{8pt} 

JavaScriptやNode.jsを中心に、オープンソースソフトウェアに
貢献してきた。それぞれの内容はGitHubで公開されている。

\subsubsection{プルリクエスト}

\begin{itemize}
  \item \url{https://github.com/socketio/socket.io/pull/2745} \\
  バージョン1.5.0でquery stringがちゃんと伝わらなかったバッグ解除のPR。
  ES5のObject.assignを使って、正しく伝わるように修正した。対応中の旧IEでは、polyfill
  を使うのが、ポイントである。

  \item \url{https://github.com/socketio/engine.io/pull/444} \\
  バージョン1.5.1のバッグ解除のPR。回帰テストを追加。

  \item \url{https://github.com/agentk/fontfacegen/pull/26} \\
  FontForgeというソフトウェアを使って、ウェブ用のフォントを出力するモジュール。
  出力されるフォントの文字選択機能を追加するPR。日本語などでフォントファイルのサイズが
  数MBにもなる場合、利用する文字だけを選択すれば、サイズを抑えるのに便利。

  \item \url{https://github.com/libxmljs/libxmljs/pull/521} \\
  XML processing instructionsに対応する、機能追加のPR。
  svgファイルにスタイルシートを追加するのに便利。
  libxmljsは裏でC言語のlibxml2に主な機能を任せているので、PRの内容の
  殆どはC++のコードで書かれている。Node.JSのV8JavaScriptエンジンの
  働き方がよく分かるようになった。

  \item \url{https://github.com/final-form/react-final-form/pull/266} \\
  Observer パターンに基づいた、Form Stateを管理するReactコンポーネント。
  Lookaheadコンポーネントをreact-final-formとreduxで使う例を追加したPRのだ。
  final-formの様々な機能の実例が提示される。

  \item \url{https://github.com/jasonpincin/stream-to-string/pull/1} \\
  Nodeのストリームを文字列に変換するモジュール。
  変換するときのencodingを指定できるよう、新しいオプションを入れた。
  回帰テストも書いた。
  PRの内容は、自分のNPMモジュール、vinyl-to-stringで利用した。

  \item \url{https://github.com/izolate/html2pug/pull/2} \\
  Expressアプリケーションで頻繁に使用されるPugを、HTMLから変換するモジュール。
  このPRの目的は、HTML Parserをjsdomからparse5に変えることで、HTMLのコメント
  やHTML Fragment等、より多くの機能に対応することだった。

  \item \url{https://github.com/itgalaxy/favicons/pull/110} \\
  faviconと関連しているファイルをプラットフォーム毎に作成するためのモジュール。
  このPRは、Gulpで使う場合、出力されるHTMLをパイプラインに入れてくれる機能を
  追加する。gulp-html2pug等、他のGulpプラグインでHTMLを更に変更可能になる。

  \item \url{https://github.com/JRJurman/rollup-plugin-polyfill/pull/2} \\
  Rollupでビルドするアプリケーションにpolyfillを注入するモジュール。
  このPRでSource Mapsの作成ができるようになった。ビルド時の警告メッセージが
  消えた。さらに、CommonJSとESMの2つのフォーマットでモジュールのインポート
  ができるようになった。
\end{itemize}

\subsubsection{NPM モジュール}

\begin{itemize}
  \item \url{https://www.npmjs.com/package/i18next-conv} \\
  継続的な貢献の結果、当モジュールの管理人となることができた。gettextのPOファイルをi18nextのJSON
  フォーマットに変換するモジュール。
  一週間で5000回以上ダウンロードされている。

  \item \url{https://www.npmjs.com/package/i18next-fetch-backend} \\
  普段i18nextで利用されるXHRバックエンドの代わりに利用できる。
  fetchを利用するから、クライアントとサーバー両方が扱うコードの場合
  に特に役立つ。
  一ヶ月で2000回以上ダウンロードされている。

  \item \url{https://www.npmjs.com/package/react-stay-scrolled} \\
  チャットウィンドウに新しいメッセージがある等の時に、スクロールダウンするのに
  役立つReactコンポーネント。
  GitHubで34個の星があり、一ヶ月で1000回以上ダウンロードされている。

  \item \url{https://www.npmjs.com/package/vinyl-contents-tostring} \\
  vinylファイル(Gulpが使うバーチャルファイル)のコンテンツを文字列に変換する
  モジュール。自作のGulpプラグインを利用している。
  一ヶ月で1000回以上ダウンロードされている。

  \item \url{https://www.npmjs.com/package/rollup-plugin-i18next-conv} \\
  Rollupでgettextファイルをi18nextのフォーマットでインポートできるモジュール。
  webviewのアプリケーション等、オフライン利用のために全てのコンテンツをバンドルする必要がある際に便利。

  \item \url{https://www.npmjs.com/package/gulp-xml-transformer} \\
  GulpでXML編集モジュール。
  一ヶ月で1000回以上ダウンロードされている。
\end{itemize}

\vspace{0.2in} 
\section{\centerline{言語}} 
\vspace{15pt}
\begin{itemize}
   \item スペイン語、ネイティブ
   \item 英語、流暢
   \item ヘブライ語、流暢
   \item 日本語、 日本語能力試験一級合格
\end{itemize}


\vspace{0.2in} 
 
\end{resume} 
\end{document}
